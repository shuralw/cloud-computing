%scrartcl: Für kürzere Ausarbeitungen. Beginnt mit section, es gibt keine chapter
%scrreprt: Für längere Ausarbeitungen (Bacherlo oder Master Thesis). Beginnt chapter
\documentclass[pdftex,a4paper,abstracton,11pt,parskip=half,bibtotocnumbered]{scrartcl} 

% Je nach LaTeX Compiler werden etwas andere Bibliotheken verwendet.
% Das Paket iftex erlaubt es, den Compiler zu überprüfen
\usepackage{iftex}

\usepackage[ngerman]{babel} % Einstellungen für den deutschen Sprachraum, neue deutsche Rechtschreibung
\ifPDFTeX
  \usepackage[utf8]{inputenc} % Umlaute erkennen. Als Option in "[] " die vom Editor verwendete Zeichencodierung auswählen
  \usepackage[T1]{fontenc}
\fi

% Für kompakter Aufzählungen
\usepackage{paralist}

\usepackage[style=ieee,backend=bibtex]{biblatex}
\addbibresource{literatur/literatur.bib}

% Für Abbildungen
\usepackage{graphicx}
\graphicspath{{./abbildungen/}}

% Ändert die Überschrift des Abstracts.
% Falls kein Abstract benötigt wird, kann die Option "abstracton" ganz oben in \documentclass entfallen.
\renewcommand{\abstractname}{Zusammenfassung}



\title{Bereitstellung einer Fullstack Anwendung auf der Google Cloud Plattform}
\author{Lukas Wessel}
\date{\today}



\begin{document}

\makeatletter
\begin{titlepage}
	\centering
	{\scshape\LARGE Fachhochschule Südwestfalen \par}
	\vspace{1cm}
%	{\scshape\Large Merkblatt\par}
	\vspace{1.5cm}
	{\huge\bfseries \@title\par}
	\vspace{3cm}
	{\Large \@author\par}
	\vspace{1cm}
	{\Large \@date\par}
	\vfill

	\raggedright
%	{\large Eingereicht bei:\par}
%	{\large Betreuer 1}
\end{titlepage}
\makeatother

\thispagestyle{empty}
\begin{abstract}
%Ein Abstrakt, also eine Kurzzusammenfassung der Arbeit ist bei einer schriftlichen Ausarbeitung nicht unbedingt notwendig.
%Bei umfangreicheren Arbeiten, also z.~B. einer Bachelor- oder Master-Thesis, sollte die Ausarbeitung in jedem Fall mit einem \textit{Abstract} beginnen.
Schriftliche Ausarbeitungen sind  wissenschaftliche Texte, die in ihrem formalen Aufbau bestimmten Richtlinien entsprechen müssen.
Dies gilt im Besonderen für Abschlussarbeiten (Bachelor- oder Masterarbeiten), aber prinzipiell auch für kürzere Aufsätze, Hausarbeiten und Projektberichte.
In diesem Leitfaden soll es darum gehen, wie Sie Ihre Ausarbeitung strukturell aufbauen sollten und welche Qualitätskriterien für die äußere und sprachliche Form gelten.
Bei einer typischen Projekt-, Bachelor- oder Masterarbeit macht die schriftliche Ausarbeitung nur einen Teil der Arbeitslast aus, ist aber gleichzeitig das wichtigste Kriterium für die Bewertung.
Daher ist es ratsam, sich möglichst frühzeitig mit den inhaltlichen und formalen Anforderungen wissenschaftlicher Texte vertraut zu machen und diese bei der Anfertigung eigener Ausarbeitungen zu berücksichtigen.
\end{abstract}

\vfill
\tableofcontents
\pagebreak

\setcounter{page}{1}

\Glossar

\section{Einleitung}


\section{Stand der Technik}
	
	\subsection{Verfügbare Technologien}
		\subsubsection{Cloud SQL}
		\subsubsection{Cloud Run}
		\subsubsection{Compute Engine}
		\subsubsection{Virtuelle Maschinen}
		\subsubsection{Docker}
		\subsubsection{Cloud Build}
		\subsubsection{GitHub Actions}
		\subsubsection{Continuous Integration}
		\subsubsection{Continuous Deployment}

\section{Methodik}
	Sollte ich hier lieber eine Aufzählung machen oder eigene Kapitel vorsehen? Wie ausführlich sollte das sein, ich will mich nicht wiederholen. 
	Bei der Umsetzung der praktischen Ausarbeitung wird einer strukturierten Vorgehensweise gefolgt, um das Ergebnis möglichst zeiteffizient zu 
	erreichen. Diese erfolgt in sechs(todo: checken obs 6 sind) Schritten.
	Als Grundlage für diverse Bereitstellungen wird Docker herangezogen. Docker verfügt über die Möglichkeit, Software-Instanzen in einer relativ kleinen 
	(bezüglich Arbeits- und physischem Speicher) Umgebung (Sandbox?) zu betreiben. Alternative Technologien hierfür sind (todo), aufgrund der 
	Bekanntheit/Nutzerbasis und der	umfangreichen Unterstützungsmöglichkeiten (Dokumentation) wird Docker an dieser Stelle als Containerisierungstechnologie
	für die vorliegende Arbeit festgelegt.  

	\subsection{Bereitstellung des Repositories auf Github}
	Vor der eigentlichen Umsetzung wurde ein Softwareprojekt umgesetzt. Der Tech-Stack umfasst eine MSSQL Datenbank, ein ASP.NET Core Backend sowie	ein
	Angular Frontend. Um es perspektivisch der Google Cloud zugänglich zu machen wurde der Code als privates GitHub Mono-Repository im Internet veröffentlicht.
	Das Projekt wird als gegeben / als Ausgangslage betrachtet und soll aus diesem Grund für die vorliegende Arbeit nicht näher betrachtet werden.  

	\subsection{Erzeugen und lokales Ausführen der Docker Images}
	Es soll verifiziert werden, dass die Docker-Images der Datenbank, des Backends und des Frontend grundsätzlich lauffähig sind. Aus diesem Grund wird 
	vorab ein Docker-Image mittels des "Docker Build" (todo: konkreter Befehl?) lokal 
	
	\subsection{Relevante Artikel der Dokumentation lesen}

	\subsection{Deployment der Datenbank}
		Die vorhandene MSSQL Datenbank soll auf der Google Cloud Plattform bereitgestellt werden. Eine Verifizierung der Funktionalität erfolgt durch 
		eine testweise Verbindung mittels eines Tools wie SQL Server Management Studio. 
	
	\subsection{Deployment des Backends}
		Das ASP.NET 5 Backend wird in diesem Schritt verfügbar gemacht. Um zu überprüfen, ob dieses abgerufen werden kann, soll die bestehende API 
		Dokumentation via Swagger getestet werden. 
	
	\subsection{Deployment des Frontends}
		Im finalen Schritt soll das Angular Frontend bereitgestellt werden. Um die erfolgreiche Umsetzung zu verifizieren soll mit einem Browser auf
		die von Google deklarierte IP-Adresse bzw. URL navigiert werden.


\begin{enumerate}
	\item Bereitstellung des  
	\item Age

\section{Umsetzung}
	Um abschätzen zu können, welche Technologien für ein Deployment in der Google Cloud in Frage kommen wurden die Dokumentation von Google herangezogen.
	Es wurde erkannt, dass derzeit die "Cloud Run" API eine für den Zweck passende Alternative darstellt. Begründen lässt sich dies darin, dass es einerseits 
	eine aktuelle Cloud-Technologie ist und andererseits von Google und mehreren Autoren in aktuellen Lektüren für diesen Anwendungszweck vorgesehen wird.(todo ZITATE)
	Darüber hinaus existiert ein begrenztes Gratis Kontingent an Ressourcen für diesen Service. So sind beispielsweise die ersten 180.000 vCPU Sekunden 
	im Monat kostenfrei. (https://cloud.google.com/run/pricing)
	Weiterhin (Fachwort möglich lol) bietet sich die Compute Engine an, da hierdurch virtuelle Maschinen und folglich Docker-Installationen ermöglicht werden.



	Grundsätzlich kamen zunächst  
	\subsection{Bereitstellung der Anwendungen}
		- Ausgangslage: Es liegt eine ASP.NET 5 

	

	\subsection{Backend}
	Grundsätzlich: Betrieb von Frontend + Backend im Docker Container, Datenbank entweder 
	Cloud Run
	/subsubsection{CORS Anpassungen}
	/subsubsection{Anpassungen der Konfigurationsdateien}
	
	\subsection{Datenbank}
		Die bisherige Umsetzung erfolgte auf einer MSSQL Datenbank. Diese liegt als Docker Image im Docker Hub vor.
		MySQL oder MariaDB, weil es hierzu Docker Images gibt. Als Alternativen können sich hier auch NoSQL Datenbanken anbieten. Für das Caching
		gleichen Anfragen und Sessions kann sich eine Redis DB anbieten, da bei jeder Anfrage, die eine gültige Session erfordert, eine Datenbankabfrage
		vonnöten ist. Für das Speichern der Buchungsdaten lohnt sich vermutlich eine dokumentenbasierte Datenbank. Der Vorteil von dokumentenbasierten 
		Datenbanken wäre eine zu erwartende Umgehung von DB-Locks bei den Schreibintensiven Operationen durch die Import Funktionalität. Im vorliegenden
		Fall wurde aufgrund der Einheitlichkeit eine relationale Datenbank gewählt. 
		
		\subsubsection{Versuch in einem Docker Container}
		Erkenntnis: Das ist grundsätzlich nicht vorgesehen, es gibt eine native Lösung: Das Cloud SQL. Es wurde zunächst versucht mit einer kostenlosen
		VM bereituzustellen. Hierbei konnte nach umfangreichen Anstrengungen und Recherchen kein Ergebnis erzielt werden. Es sei an dieser Stelle jedoch
		kritisch zu vermerken, dass die Expertise des Autors im Netzwerkbereich nicht umfangreich genug waren, um eine korrekte Lösung zu implementieren.
		Bei ausreichenden Kenntnissen besteht die Möglichkeit, dass hier dennoch eine korrekte Umsetzung erfolgen kann. Google sieht vor, dass man die
		Cloud-native SQL Lösung "Cloud SQL" verwendet.(todo: belegen, wenn das nicht geht dann sollte das irgendwie umformuliert werden oder raus.)
		Die Verbindung eines Cloud Run Service mit einer Cloud SQL Datenbank liegt umfangreich von Google dokumentiert vor. 

		\subsubsection{Netzwerkprobleme}
		\subsubsection{CloudSQL}
		\subsubsection{Alternative Möglichkeiten}
	\subsection{Frontend} (ggf. zusammen mit Backend?)

	(\subsection{Aufgetretene Probleme}) vermutlich eher beiläufig

\section{Evaluation und Ergebnisse}
	Stärken und Schwächen herausstellen
	Verbesserungsfähige Aspekte benennen 
		- App.Config auslagern in Environment Variables
	Lösungsansätze aufzeigen

\section{Zusammenfassung}

\section{Ausblick}

%\appendix
%
\section{Der erste Anhang}

\subsection{Mit einem Unterabschnitt}



% Bibliographie
\printbibliography

\end{document}
