%scrartcl: Für kürzere Ausarbeitungen. Beginnt mit section, es gibt keine chapter
%scrreprt: Für längere Ausarbeitungen (Bacherlo oder Master Thesis). Beginnt chapter
\documentclass[pdftex,a4paper,abstracton,11pt,parskip=half,bibtotocnumbered]{scrartcl} 

% Je nach LaTeX Compiler werden etwas andere Bibliotheken verwendet.
% Das Paket iftex erlaubt es, den Compiler zu überprüfen
\usepackage{iftex}

\usepackage[ngerman]{babel} % Einstellungen für den deutschen Sprachraum, neue deutsche Rechtschreibung
\ifPDFTeX
  \usepackage[utf8]{inputenc} % Umlaute erkennen. Als Option in "[] " die vom Editor verwendete Zeichencodierung auswählen
  \usepackage[T1]{fontenc}
\fi

\usepackage{cite}

% Für kompakter Aufzählungen
\usepackage{paralist}

\usepackage[style=ieee,backend=bibtex]{biblatex}
\addbibresource{literatur/literatur.bib}

% Für Abbildungen
\usepackage{graphicx}
\graphicspath{{./abbildungen/}}

% Ändert die Überschrift des Abstracts.
% Falls kein Abstract benötigt wird, kann die Option "abstracton" ganz oben in \documentclass entfallen.
\renewcommand{\abstractname}{Zusammenfassung}



\title{Kontinuierliche Bereitstellung einer Fullstack Anwendung auf der Google Cloud Plattform}
\author{Lukas Wessel}
\date{\today}



\begin{document}

\makeatletter
\begin{titlepage}
	\centering
	{\scshape\LARGE Fachhochschule Südwestfalen \par}
	\vspace{1cm}
%	{\scshape\Large Merkblatt\par}
	\vspace{1.5cm}
	{\huge\bfseries \@title\par}
	\vspace{3cm}
	{\Large \@author\par}
	\vspace{1cm}
	{\Large \@date\par}
	\vfill

	\raggedright
%	{\large Eingereicht bei:\par}
%	{\large Betreuer 1}
\end{titlepage}
\makeatother

\thispagestyle{empty}
\begin{abstract}
%Ein Abstrakt, also eine Kurzzusammenfassung der Arbeit ist bei einer schriftlichen Ausarbeitung nicht unbedingt notwendig.
%Bei umfangreicheren Arbeiten, also z.~B. einer Bachelor- oder Master-Thesis, sollte die Ausarbeitung in jedem Fall mit einem \textit{Abstract} beginnen.
Schriftliche Ausarbeitungen sind wissenschaftliche Texte, die in ihrem formalen Aufbau bestimmten Richtlinien entsprechen müssen.
Dies gilt im Besonderen für Abschlussarbeiten (Bachelor- oder Masterarbeiten), aber prinzipiell auch für kürzere Aufsätze, Hausarbeiten und Projektberichte.
In diesem Leitfaden soll es darum gehen, wie Sie Ihre Ausarbeitung strukturell aufbauen sollten und welche Qualitätskriterien für die äußere und sprachliche Form gelten.
Bei einer typischen Projekt-, Bachelor- oder Masterarbeit macht die schriftliche Ausarbeitung nur einen Teil der Arbeitslast aus, ist aber gleichzeitig das wichtigste Kriterium für die Bewertung.
Daher ist es ratsam, sich möglichst frühzeitig mit den inhaltlichen und formalen Anforderungen wissenschaftlicher Texte vertraut zu machen und diese bei der Anfertigung eigener Ausarbeitungen zu berücksichtigen.
\end{abstract}

\vfill
\tableofcontents
\pagebreak

\setcounter{page}{1}

\section{Glossar}
DevOps \\
Mandantentrennung - Personenbezogene Daten dürfen keinem fremden Kunden zugänglich werden. Aus diesem Grund müssen die Daten der Kunden voneinander getrennt werden, 
man spricht in diesem Fall von Mandantentrennung.\\
Serverless

\section{Einleitung}
 - Welche Kombination an Cloud-Services bietet sich zurzeit für eine Fullstackanwendung an?
 - Welche Alternativen Technologien können verwendet werden?


\section{Stand der Technik}

	\subsection{Verfügbare Technologien} 
	(todo: Ich brauche auch 3.2)
	(todo: Soll ich die Kapitelüberschrift streichen? Ist das doppelt mit Stand der Technik? Ich find das ist irgendwie doppelt.)

	In diesem Kapitel werden die in Frage kommenden Technologien und Konzepte hinsichtlich des Betriebs, der Bereitstellung der Images und des
	CI/CD Prozesses beleuchtet. 

		\subsubsection{Cloud SQL}		
			Cloud SQL ist ein Service zum Bereitstellen von Datenbanken, die von Google in dessen Cloud Plattform bereitgestellt wird. Als Anwender kann hier eine
			PostGreSQL, MySQL oder eine MSSQL Datenbank instanziiert werden. Hierbei werden laut Ożarowska die Daten "stored and processed in the cloud, on the infrastructure of a cloud 
			service provider". \cite{ożarowska_2020}

		\subsubsection{Cloud Run}
			Venema definiert Cloud Run wie folgt: "Cloud Run is a platform on Google Cloud that lets you build scalable and reliable web-based applications. As
 			a developer, you can get very close to being able to just write your code and push it, and then let the platform deploy, run, and scale your application for you." \cite[S.1]{Venema_2021} \\
			Hierbei können laut Venema sehr schnell 1000 Container erzeugt werden. \cite[vgl.][S.XVI]{Venema_2021}
			Die Applikationen lassen sich in Netzwerken vereinen oder voneinander trennen, mit Rechten und Rollen ausstatten (IAM) und mit verschiedensten 
			Datenbankservices wie etwa Cloud SQL oder Firestore verbinden. Für Cloud Run sind Containertechnologien wie Docker vorgesehen, um deren Images, 
			welche eine "pre-built app" \cite{johnson_2020} beinhalten, zu "pullen" und zu instanziieren. Die Technologie lässt sich effizienter nutzen, indem sie mit "Continuous 
			Deployment" kombiniert wird (siehe Kapitel 3.1.10), da hierdurch der administrative Aufwand von zukünftigen Bereitstellungen reduziert wird \cite{lamm_2021}.
			(todo: längerer Abschnitt)
			(todo: KNative miteinbringen)

		\subsubsection{Virtuelle Maschinen}
			Lackes und Siepermann definieren im "Gabler Wirtschaftslexikon" eine virtuelle Maschine wie folgt: "Bei Mehrbenutzerbetrieb, v.a. bei Teilnehmerbetrieb, wird durch
			das Betriebssystem für jeden einzelnen Teilnehmer eine eigene Hardware-Umgebung simuliert, eine „virtuelle Maschine”. Diese enthält z.B. einen eigenen Arbeitsspeicher, 
			Magnetplattenspeicher und Drucker, individuell für den Teilnehmer. Prinzipiell kann jede virtuelle Maschine mit einem anderen Betriebssystem betrieben
			werden. Die interne Realisierung der virtuellen Maschine durch das Betriebssystem erfolgt natürlich auf den realen Geräten; z.B. wird der virtuelle Drucker
			eines Teilnehmers auf einem realen Drucker der Datenverarbeitungsanlage abgebildet." \cite{lackes_siepermann_2018} \\
			Im Kontext einer Cloud spielen diese eine wichtige Rolle, da sie Mandantentrennung unter möglichst effizienter Nutzung der Hardwareressourcen
			des Rechenzentrums gewährleisten können.
		
		\subsubsection{Compute Engine}
			Virtuelle Maschinen werden auf der Google Cloud Plattform mit der Compute Engine erzeugt. Hierbei entsteht ein großer administrativer Aufwand, 
			da die Installation von Updates und Installationen wie Docker oder Datenbanktechnologien durch den Kunden vorgenommen werden. 
			Auf der anderen Hand liegt eine hohe Kontrolle des Servers vor, da viele Einstellungen des Servers vom Kunden vorgenommen werden können. Der Kunde
			genießt somit große Freiheit in der Wahl seiner Software und Einstellungen, ist jedoch auch verantwortlich für die Pflege hierfür.
	
		\subsubsection{Docker}
			Die Softwareumgebung vom Entwickler sollte vom Administrator replizierbar sein, damit keine Probleme aufgrund der Umgebung in Test- oder Produktivumgebungen 
			auftreten. Docker bietet hierfür eine Lösung, es ist die allgegenwärtige Technologie \cite[vgl.][S.40]{Venema_2021} um Software in Containern zu verpacken (containerisieren), 
			auf einer zentralen Plattform (Registry) hochzuladen (Push) und diese wiederum herunterzuladen (Pull). Es virtualisiert laut Clancy im Gegensatz zu virtuellen Maschinen
			nicht die Hardware, sondern das Betriebssystem. \cite[vgl.][]{clancy_2021} Es erlaubt einen höheren Nutzungsgrad einer (virtuellen) Maschine, da mehrere 
			kleine Container-Instanzen hierauf gleichzeitig betrieben werden können.

		\subsubsection{Cloud Build}
			Grundsätzlich werden Docker Images vom Entwickler (todo Beleg) erzeugt. Diese können dann einem Administrator zur Verfügung gestellt werden, der 
			diese dann auf einem Server publiziert. Dieses Verfahren kann jedoch durch eine simplere Vorgehensweise ersetzt werden. Cloud Build ersetzt den Schritt der Image 
			Erzeugung insoweit, alsdass man diesen Service mit einem Code Repository und eine Konfigurationsdatei wie beispielsweise ein Dockerfile verknüpft. 
			Sodann ist es möglich mittels eines Kommandos eine neue Version basierend auf dem aktuellen Code-Stand zu kompilieren. Das Erzeugnis kann 
			anschließend auf einer Registry wie beispielsweise der Artifact Registry oder der Container Registry von Google gepusht werden. 

		\subsubsection{GitHub Actions}
			Es besteht die Möglichkeit anstelle von Google Cloud Build auch GitHub Actions zu verwenden. Hiermit können Images ebenfalls in der Cloud gebaut werden. 
			Sofern man sich für diese Lösung entscheidet vermeidet man für diese Technologie einen "Vendor Lock-In", was eine Abhängigkeit von einem Anbieter beschreibt.

		\subsubsection{Artifact Registry}
			Die Artifact Registry (AR) von Google ist ein Ablageort für Container Images und weitere "Build-Artefakte". Diese löst die hierfür zuvor vorgesehene Container Registry ab.
			Artefakte können Docker Images oder PaketeBuild-Abhängigkeiten sein. \cite[vgl.][]{martinez_2020} 

		\subsubsection{Docker Registry}
			Die Docker Registry ist eine Container Image Registry vom Unternehmen Docker. Wie auch die GitHub Actions vermeidet man durch die Verwendung der Docker Registry 
			einen Vendor Lock-In, da sie eine Alternative zur Artifact Registry darstellt.
			
		\subsubsection{Continuous Integration}
			Continuous Integration bezeichnet ein Prinzip, bei dem Quellcode kontinuierlich in eine Codebasis integriert wird. Hierbei gilt laut Meyer eine
			"tägliche" Basis als kontinuierlich.\cite[vgl.][]{Meyer_2014}
			Eine Integrierung von Code gilt als Zusammenführung von Entwickler-spezifischen Codeänderungen in einen kollaborativen Codestand, was für 
			gewöhnlich als main- , beziehungsweise master-Branch bezeichnet wird. Darüber hinaus umfasst die kontinuierliche Integration von Code auch 
			einen möglichst konstant lauffähigen Build laut Meyer.\cite[vgl.][]{Meyer_2014} Dieser sollte nach jeder Zusammenführung von Code stattfinden
			und erfolgreich sein.

		\subsubsection{Continuous Deployment}
			Continuous Deployment liegt vor, wenn die Integration von Code ein Deployment hervorruft. Exakt wie bei Continuous Integration sollten diese Deployments
			häufig und in kurzen Abständen stattfinden. Es beschreibt laut Quibeldey-Cirkel und Thelen "solche Mikro-Release-Zyklen, bei denen jede geänderte
			Programmzeile zur Auslieferung einer neuen, vollständigen und qualitätsgesicherten Version [führt]."
			
\section{Methodik}
	Sollte ich hier lieber eine Aufzählung machen oder eigene Kapitel vorsehen? Wie ausführlich sollte das sein, ich will mich nicht wiederholen. 
	Bei der Umsetzung der praktischen Ausarbeitung wird einer strukturierten Vorgehensweise gefolgt, um das Ergebnis möglichst zeiteffizient zu 
	erreichen. Diese erfolgt in sechs(todo: checken obs 6 sind) Schritten.
	Als Grundlage für diverse Bereitstellungen wird Docker herangezogen. Docker verfügt über die Möglichkeit, Software-Instanzen in einer relativ kleinen 
	(bezüglich Arbeits- und physischem Speicher) Umgebung (Sandbox?) zu betreiben. Alternative Technologien hierfür sind (todo), aufgrund der 
	Bekanntheit/Nutzerbasis und der	umfangreichen Unterstützungsmöglichkeiten (Dokumentation) wird Docker an dieser Stelle als Containerisierungstechnologie
	für die vorliegende Arbeit festgelegt.  

	\subsection{Bereitstellung des Repositories auf Github}
	Vor der eigentlichen Umsetzung wurde ein Softwareprojekt umgesetzt. Der Tech-Stack umfasst eine MSSQL Datenbank, ein ASP.NET Core Backend sowie	ein
	Angular Frontend. Um es perspektivisch der Google Cloud zugänglich zu machen wurde der Code als privates GitHub Mono-Repository im Internet veröffentlicht.
	Das Projekt wird als gegeben / als Ausgangslage betrachtet und soll aus diesem Grund für die vorliegende Arbeit nicht näher betrachtet werden.  

	\subsection{Erzeugen und lokales Ausführen der Docker Images}
	Es soll verifiziert werden, dass die Docker-Images der Datenbank, des Backends und des Frontend grundsätzlich lauffähig sind. Aus diesem Grund wird 
	vorab ein Docker-Image mittels des "Docker Build" (todo: konkreter Befehl?) lokal erzeugt. Eine mögliche lokale Ausführung einer serverlosen Anwendung
	ist laut Lasczczak "critical for efficiant development".

	\subsection{Relevante Artikel der Dokumentation lesen}
	Um zu erlernen, welche Google Services sich grundsätzlich für den Zweck eignen und wie diese miteinander verkettet werden können soll die Google
	Cloud Dokumentation hinsichtlich der genannten Cloud Technologien nachvollzogen werden.

	\subsection{Build der Anwendung und Publikation in einer Registry}
	Der Code soll vom GitHub Repository auf der Google Cloud Plattform gebaut werden und anschließend in einer Image Registry für den weiteren Prozess
	abgelegt werden. Hierfür sind potenziell die Google Cloud Docker Image Registry oder der Docker Hub geeignet, sie erlauben das "Pushing" und "Pulling"
	von Docker Images. 

	\subsection{Deployment der Datenbank}
	Die vorhandene MSSQL Datenbank soll auf der Google Cloud Plattform bereitgestellt werden. Eine Verifizierung der Funktionalität erfolgt durch 
	eine testweise Verbindung mittels eines Tools wie SQL Server Management Studio. 

	\subsection{Deployment des Backends}
	Das ASP.NET 5 Backend wird in diesem Schritt verfügbar gemacht. Um zu überprüfen, ob dieses abgerufen werden kann, soll die bestehende API 
	Dokumentation via Swagger getestet werden. 
	
	\subsection{Deployment des Frontends}
	Im finalen Schritt soll das Angular Frontend bereitgestellt werden. Um die erfolgreiche Umsetzung zu verifizieren soll mit einem Browser auf
	die von Google deklarierte IP-Adresse bzw. URL navigiert werden.


\begin{enumerate}
	\item Bereitstellung des  
	\item Age
\end{enumerate}

\section{Umsetzung}
	Um abschätzen zu können, welche Technologien für ein Deployment in der Google Cloud in Frage kommen wurde die Dokumentation von Google herangezogen.
	Es wurde erkannt, dass derzeit die "Cloud Run" API eine für den Betrieb einer webbasierten Fullstack Anwendung passende Alternative darstellt. Begründen lässt sich dies darin,
	dass es einerseits eine aktuelle Cloud-Technologie ist und andererseits durch dessen serverlosen Ausprägung von Google und mehreren Autoren in aktuellen Lektüren
	für diesen Anwendungszweck	vorgesehen wird. \cites[vgl.][]{daly_2022}{laszczak_2020}
	Darüber hinaus existiert ein begrenztes Gratis Kontingent an Ressourcen für diesen Service. So sind beispielsweise die ersten 180.000 vCPU Sekunden 
	im Monat kostenfrei. (https://cloud.google.com/run/pricing)
	Weiterhin (Fachwort möglich lol) eignet sich grundsätzlich auch die Compute Engine für den Betrieb der Applikationen, da hierdurch virtuelle Maschinen
	und folglich Docker-Installationen ermöglicht werden.

	\subsection{Cloud Build}
	Der Google Cloud Build wurde verwendet, um den Code von GitHub in Docker Images zu transformieren und diese anschließend in einer Container Registry 
	abzulegen. Da Docker verwendet werden soll, wird während des Erstellprozesses "Docker" als Format selektiert und als Quelle wurde "GitHub 
	(Cloud Build-GitHub-Anwendung)" verwendet, um den Build Prozess mit dem GitHub Repository zu verbinden. Im Anschluss muss eine Authentifizierung für dieses
	Repository durchgeführt werden. Hierbei fordert Google die erforderlichen Berechtigungen an, sodass keine weitere Maßnahmen auf GitHub erforderlich sind. 
	Es werden diverse Events wie "Push to Branch" oder "Pull Request" zum initiieren des Builds angeboten. Die Entscheidung hierfür sollte auf den
	jeweiligen Team- beziehungsweise Unternehmensrichtlinien basieren, da dies eine kollektiv entschiedene Konvention sein sollte.
	Abschließend muss noch eine Konfiguration angegeben, in der Ausarbeitung wurde hierfür ein Dockerfile verwendet.\\
	Sofern man große Projekte oder Software mit umfangreichen Abhängigkeiten bauen möchte kann es vorkommen, dass Zeitüberschreitungen auftreten.  
	Im weiteren Verlauf der Arbeit wird ersichtlich, dass man für ein Continuous Deployment einen speziellen Build Prozess erzeugen muss. (todo: Validieren)

	
	(todo: Verschieben in anderes Kapitel) Es ist vonnöten, dass man eine YAML Datei schreibt und diese dem Build Prozess zur Verfügung anfügt. Optimalerweise wird diese auch dem Repository 
	hinzugefügt, da sie genau wie reguläre Code Dateien auch einen Veränderungsprozess durchlaufen. (todo: Quelle)

	\subsection{Docker Registry}
	Es wurde während des Lesens der Dokumentation erkannt, dass es eine aktuellere Technologie gibt, die Google für Docker Registries vorsieht.
	Obgleich der Name "Google Container Registry" suggeriert, dass dieser Dienst für das Ablegen von Images verwendet werden sollte, empfiehlt die Google
	Dokumentation die Verwendung der aktuelleren "Artifact Registry". Martinez formuliert diesbezüglich: "[The Google Cloud Platform Artifact 
	Registry] will [..] replace our current [Google Cloud Registry] in the future [...]" \cite{martinez_2020} 
	(todo: Sollte ich die Artifact Registry auch noch in den Werkzeugen ausformulieren? Ja oder?)

	Grundsätzlich kamen zunächst  
	\subsection{Bereitstellung der Anwendungen}
		- Ausgangslage: Es liegt eine ASP.NET 5 
	

	\subsection{Backend}
	Grundsätzlich: Betrieb von Frontend + Backend im Docker Container, Datenbank entweder 
	Cloud Run
	\subsubsection{CORS Anpassungen}
	\subsubsection{Anpassungen der Konfigurationsdateien}
	
	\subsection{Datenbank mittels Compute Engine}
		Die Bereitstellung der Datenbank erfolgte zunächst auf einer Compute Engine Instanz. Die erstellte virtuelle Maschine wurde hierbei mit einer Debian und einer Docker Installation versehen. Da im Softwareprojekt die Datenbank mit einer MSSQL Datenbank umgesetzt wurde, sollte auch der Betrieb mit einer MSSQL Datenbank stattfinden. Um dies zu realisieren wurde das offizielle Microsoft SQL Server Docker Image vom Docker Hub instanziiert. Dies lies sich mit dem Befehl "" bewerkstelligen. Damit man sich anschließend mit der Datenbank durch ein Client Tool wie SQL Server Management Studio verbinden konnte musste noch der Port 3306 in der Google Cloud Firewall für die erstellte virtuelle Maschine freigegeben werden. Sodann war es möglich, sich mit den festgelegten Benutzerdaten und der externen IP Adresse sowie dem Port gegen die Datenbank zu verbinden. Die Notation der Adresse erfolgt hierbei grundsätzlich im Format "ip,port". Sowohl das Schema als auch die Daten ließen sich nun via Insert Skript einspielen. //
		
		
		Je nach Technologie Stack kommen auch andere Datenbanken in Frage. Sofern es ein geeignetes Docker Image hierfür gibt, kann diese Datenbank ebenfalls hiermit betrieben werden. (todo: Zeitform?) Da in dem zugrunde liegenden Softwareprojekt bereits eine MSSQL Datenbank vorlag, wurde auch ebenjene für diese Ausarbeitung verwendet. Als Alternativen würden sich für das vorliegende Szenario entweder andere relationale Datenbanken anbieten. Sofern die Anwendung eine höhere horizontale Skalierbarkeit bieten soll, kann eine NoSQL Technologie in Betracht gezogen werden. Für das Projekt "Finanzübersicht" kommen dokumentenbasierte Datenbanken für die Buchungsdaten in Frage. Dieses könnte insbesondere Lastspitzen bei der Verwendung der Import-Funktion von Buchungsdaten relativieren. Bei relationalen Datenbanken besteht hier die Gefahr, dass bei vielen gleichzeitigen Schreibvorgängen auf dieser Tabelle viele Datenbank Locks entstehen würden. [Die Folge hiervon wäre der potenzielle Stillstand der Datenbank]. Für Session-Daten und Caching kann eine RedisDB sinnvoll sein (todo: zitat), da...(todo: Will ich diesen Absatz behalten?) Sofern weiterhin ein relationales Modell geführt werden sollte kann auch der Google Cloud Spanner eine Option sein, da .... 
		
		Den grundsätzlichen Betrieb einer Datenbank in einem Container gilt es jedoch zu hinterfragen. Laut Heddings ist Docker nicht für "stateful services"\cite[vgl.][]{heddings_2020} wie Datenbanken vorgesehen. Sämtliche Daten sind laut ihm flüchtig und werden vernichtet, sobald der Container gelöscht wird. Dieses Problem lässt sich jedoch gemäß Heddings in gewissem Maße umgehen, indem Docker Volumes für die Persistierung der Daten verwendet werden.\cite[vgl.][]{heddings_2020}
		
		
	
		Erkenntnis: Das ist grundsätzlich nicht vorgesehen, es gibt eine native Lösung: Das Cloud SQL. Es wurde zunächst versucht mit einer kostenlosen
		VM bereituzustellen. Hierbei konnte nach umfangreichen Anstrengungen und Recherchen kein Ergebnis erzielt werden. Es sei an dieser Stelle jedoch
		kritisch zu vermerken, dass die Expertise des Autors im Netzwerkbereich nicht umfangreich genug war, um eine potenziell korrekte Lösung zu implementieren.
		Bei ausreichenden Kenntnissen besteht die Möglichkeit, dass hier dennoch eine adäquate Umsetzung erfolgen kann. Google sieht vor, dass man die
		Cloud-native SQL Lösung "Cloud SQL" verwendet.(todo: belegen, wenn das nicht geht dann sollte das irgendwie umformuliert werden oder raus.)
		Die Verbindung eines Cloud Run Service mit einer Cloud SQL Datenbank liegt umfangreich von Google dokumentiert vor. 

		\subsubsection{Netzwerkprobleme}
		
		\subsubsection{CloudSQL}
		
		\subsubsection{Alternative Möglichkeiten}
	\subsection{Frontend} (ggf. zusammen mit Backend?)

	(\subsection{Aufgetretene Probleme}) vermutlich eher beiläufig

Falls ich ne kritische Betrachtung mache: Großes Problem ist der Vendor Lock-In



\section{Evaluation und Ergebnisse}
	Stärken und Schwächen herausstellen
	Verbesserungsfähige Aspekte benennen 
		- App.Config auslagern in Environment Variables
	Lösungsansätze aufzeigen

	Es wurden keine Sicherheitsaspekte betrachtet. In Unternehmenslösungen sollten beispielsweise Rechte- und Rollenkonzepte erarbeitet und angewandt werden.
\section{Zusammenfassung}

\section{Ausblick}

%\appendix
%
\section{Der erste Anhang}

\subsection{Mit einem Unterabschnitt}



% Bibliographie
\printbibliography

\end{document}
