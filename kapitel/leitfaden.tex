%Bei einem Report (documentclass "scrreprt") oder einem Buch ("scrbook")  ist das 

\section{Allgemeines}

Im Laufe Ihres Studiums werden Sie verschiedene schriftliche Ausarbeitungen anfertigen.
Die Abschlussarbeit ist dabei sicher die Ausarbeitung mit dem größten Umfang, aber auch kleinere Arbeiten, wie etwa Projektberichte und Hausarbeiten, sind Ausarbeitungen, für die ähnliche Maßgaben gelten.
In diesem Dokument möchte ich Ihnen ein paar Hinweise und Tipps auf den Weg geben, die Sie beim Schreiben von Ausarbeitungen beachten sollten.

In den technischen Disziplinen, zu denen die Ingenieurwissenschaften und auch die Informatik zählen, liegen den Ausarbeitungen häufig technische Implementierungen zugrunde.
In der Informatik ist dies normalerweise Software, die im Rahmen der Arbeit entsteht.
Oft ist gerade dieser praktische Teil besonders interessant und man vertieft sich darin,  das Verfassen der Ausarbeitung hingegen wird gelegentlich unterschätzt.
Dabei sollte man sich immer darüber im Klaren sein, dass die Ausarbeitung am Ende des Projekts die wichtigste Grundlage für die Bewertung darstellt.

Jede Ausarbeitung stellt für sich eine wissenschaftliche Arbeit dar und unterliegt den dafür üblichen Regeln.
Bei einer wissenschaftlichen Arbeit muss der Weg zum Erkenntnisfortschritt nachvollziehbar sein und daher reicht die reine Beschreibung der (im Zuge der Arbeit entstandenen) Implementierung nicht aus. 
Es müssen auch die verwendeten Konzepte, Daten und Werkzeuge eingeführt und erläutert werden.
Die Zielsetzung muss klar formuliert und die gewählten Methoden argumentativ begründet werden.
Dabei muss immer klar sein, welche Leistung im Rahmen Ihrer Arbeit entstanden ist und was Sie von welcher Quelle übernommen haben.
Die erzielten Ergebnisse sollten immer kritisch bewertet werden. 
Hierbei gilt es, die Stärken der eignen Arbeit herauszustellen, aber auch verbesserungsfähige Aspekte zu benennen und gegebenenfalls Lösungsansätze aufzuzeigen.

Neben den wissenschaftlichen Ansprüchen sind auch die Gliederung sowie die sprachliche Qualität wichtige Entscheidungskriterien für die Bewertung.
Kann man einer Ausarbeitung, aufgrund eines schlechten Aufbaus, vieler orthographischer oder grammatikalischer Fehler und Ausdrucksschwächen nicht folgen, so kann auch die  technisch-wissenschaftliche Leistung \textit{hinter} der Ausarbeitung nicht klar werden.
Damit Ihre technische Leistung auch angemessen bewertet werden kann, sollten Sie daher entsprechend Wert auf den Inhalt und die Form Ihrer schriftlichen Ausarbeitung legen.

\section{Aufbau der Arbeit}

In diesem Abschnitt werden Fragen besprochen, die sich mit der Struktur und dem logischen Aufbau der Ausarbeitung befassen.

\subsection{Der \textit{rote Faden}}

Wissenschaftliche Texte dienen der Darstellung, Vermittlung und kritischen Diskussion eines Sachproblems.
Um den Text möglichst interessant zu gestalten, sollten Sie sich stets vor Auge führen, \textbf{an wen} sich der Text wendet, und \textbf{welche Fragen} sich der Leser zum aktuellen Zeitpunkt stellt.
Wenn Sie es schaffen, die Fragen des Lesers \textit{vorherzusehen}, können Sie an entsprechender Stelle die notwendigen Informationen oder Argumente liefern.
So erfüllen Sie die Erwartungshaltung des Lesers und ermutigen ihn, interessiert weiterzulesen.

Beginnen wir mit der ersten Leitfrage: An wen richtet sich Ihre Ausarbeitung?
Hierzu sollte vor allem klargestellt werden, das die Zielgruppe der Arbeit nicht (nur) die Betreuer sind!
Inhaltliche Aspekte, die Sie möglicherweise mit Ihren Betreuern mündlich oder schriftlich besprochen haben, dürfen Sie deswegen keinesfalls weglassen.
Der Text muss für sich allein verständlich sein und daher sollte alles, was inhaltlich wichtig ist, auch in der Ausarbeitung genannt werden.

Allgemein kann man sagen, dass die Ausarbeitung so geschrieben sein sollte, dass fachlich qualifizierte Personen der Arbeit folgen können.
Nun ist die Frage, wer ist \textit{fachlich qualifiziert} bzw. welches fachliche Wissen können Sie vom Leser voraussetzen.
Eine gute Faustregel ist hierbei, anzunehmen, dass der Leser den gleichen Kenntnisstand besitzt, den Sie \textbf{vor Beginn} der Bearbeitung des Themas hatten.
Dies gilt natürlich nicht für relevantes \textit{Spezialwissen}, das Sie vielleicht schon vor der Bearbeitung besaßen.
Den Kenntnisstand Ihrer Leser richtig einzuschätzen, ist sehr wichtig für die Qualität der Arbeit!
Wählen Sie ein zu hohes Niveau, wird Ihr Text unverständlich und Sie hängen den Leser ab, ist das Niveau zu niedrig, laufen Sie Gefahr den Leser zu langweilen.

Die zweite Leitfrage ist: Welche Fragen stellt sich der Leser gerade?
Die Antwort darauf hängt natürlich stark von Ihrem Thema ab und davon an welcher Stelle sich der Leser im Text befindet.
Es gibt einige sehr generelle Fragestellungen, die man ganz zu Beginn der Arbeit beantworten sollte, um überhaupt das Interesse des Lesers zu wecken:
\begin{enumerate}
\item Worum geht es in der Arbeit? Was ist die Motivation, dieses Thema zu behandeln?
\item Was waren die Ziele der Arbeit? Was wurde erreicht? Was sind die Hauptleistungen der Arbeit?
\item Wie wurde das Ergebnis erreicht? Welche Methoden wurden eingesetzt?
\item Kann ich (der Leser) das Ergebnis für meine Zwecke verwenden?
\end{enumerate}

Aus diesen Fragen sollten Sie bei jeder Ausarbeitung auf den ersten ein bis zwei Seiten eingehen.
Bei längeren Texten werden diese Fragen im Abstract, also der Kurzzusammenfassung, die vor dem eigentlichen Text steht, behandelt und im Verlauf des ersten Kapitels nochmals aufgegriffen.
Bei kurzen Ausarbeitungen kann der Abstract entfallen, die oben genannten Punkte werden dann in der Einleitung behandelt.

\paragraph{Die Motivation der Arbeit}

Zu Beginn der Ausarbeitung ist es besonders wichtig, dem Leser zu erklären, warum Sie das Thema bearbeiten.
Dabei ist nicht unbedingt Ihre eigene Motivation gemeint, sondern vielmehr die Frage, warum die Problemstellung Ihrer Arbeit relevant ist.
Angemessene Motivationsgründe sind etwa:
\begin{compactitem}
\item Sie lösen ein Problem, für das es bisher keine, oder keine gute Lösung gibt.
\item Ein Unternehmen kann Ihre Lösung einsetzen, um damit Profit zu erwirtschaften.
\item Sie vergleichen verschiedene Produkte oder Methoden, um damit die Entscheidungsfindung bei der Auswahl zu erleichtern.
\item Sie stellen ein komplexes Thema für eine bestimmte Zielgruppe angemessen dar.
\end{compactitem}


\subsection{Die Gliederung}
\label{sec:gliederung}

Eine Ausarbeitung gliedert sich immer in Einleitung, Hauptteil, Schlussteil sowie eventuelle Anhänge.
Die Einleitung führt in das Thema ein, beschreibt die Motivation und nennt die konkreten Ziele der Arbeit.
Sofern es eine mit dem Betreuer abgesprochene Aufgabenstellung gibt, sollte man diese hier möglichst genau wiedergeben.
Dabei ist es übrigens nicht tragisch, wenn ein vorab gesetztes Ziel im Rahmen der Arbeit nicht erreicht wurde.
Solange Sie triftige (d.~h. üblicherweise technische) Gründe nennen können, warum das Ziel nicht erreicht wurde, ist auch das ein Ergebnis.
Bei einer wissenschaftlichen Arbeit sollte in der Einleitung auch der \textit{wissenschaftliche Beitrag} (engl. \textit{scientific contribution}) klar formuliert werden.
Gerade bei publizierten Arbeiten ist darüber genau auszumachen, welche neue Erkenntnis durch die vorliegende Arbeit entstanden ist.
Den Abschluss der Einleitung bildet meistens ein Überblick über die folgenden Kapitel der Ausarbeitung. 
Gerade bei längeren Arbeiten ist diese Übersicht für den Leser wichtig.

Der Hauptteil bildet den Kern der Arbeit und ist vom Umfang her der mit Abstand umfangreichste Teil der Ausarbeitung.
Üblicherweise besteht er  aus mehreren Kapiteln oder Abschnitten, die häufig einem festen Aufbau folgen:
\begin{enumerate}
\item Theoretischer Teil: In diesem Teil werden die Grundlagen des Themas eingeführt und erklärt. Das entsprechende Kapitel heißt folglich meist \textit{Grundlagen} oder \textit{Stand der Technik}. 
Wenn Sie für Ihren technischen Teil bestimmte Methoden oder Werkzeuge einsetzen, sollten Sie diese hier einzeln beschreiben.
Dabei gilt es, die jeweiligen Grundlagen gerade so genau zu beschreiben, wie es für das Verständnis der Arbeit notwendig ist.
Damit der Leser die Gelegenheit hat, Genaueres zu den Grundlagen zu erfahren, sollten Sie hier gute Referenzen angeben (vgl. Abschnitt~\ref{sec:zitieren}).
Sind für die Arbeit sehr verschiedene Grundlagen erforderlich, kann der theoretische Teil auch in mehrere Kapitel aufgeteilt werden. 
\item Praktischer Teil: Im praktischen Teil wird Ihre \textit{Lösung} der Problemstellung beschrieben. Häufig ist dieser Teil weiter untergliedert, z.~B. in einen Abschnitt zu Ihrer Methodik und weitere zur konkreten Ausführung bzw. Implementierung.

Der Teil zur Methodik hilft dem Leser, Ihre Vorgehensweise besser zu überblicken und die einzelnen Teile im Zusammenhang zu sehen.
Bei der folgenden Beschreibung Ihrer konkreten Umsetzung sollten Sie sorgfältig abwägen, wie detailliert Sie Ihre Lösung präsentieren.
Seitenweise Quelltext abzudrucken ist beispielsweise in aller Regel nicht sehr hilfreich.
Wählen Sie besser die zentralen Elemente Ihrer technischen Umsetzung aus und beschreiben Sie nur diese im Detail.
Sehr hilfreich für das Verständnis sind im praktischen Teil auch repräsentative Beispiele anhand derer Sie bestimmte Abläufe oder Funktionen erklären.

Den Abschluss des technischen Teils bilden üblicherweise ein oder mehrere Abschnitte mit den Ergebnissen.
Diese Informationen sind für den Leser zentral, um bewerten zu können, \textit{wie gut} Ihre Ziele erreicht wurden.
Leider fällt dieser Teil in vielen Ausarbeitungen deutlich zu knapp aus, z.~B. weil Studierende mit dem Fertigstellen der Implementierung den gesamten Praxisteil als abgeschlossen betrachten.
Besser ist es, wenn Sie nach Testfällen oder Eingabedaten suchen, mit denen Sie Ihre Lösung qualitativ bewerten können.
Die Resultate dieser Analysen liefern wichtige Inhalte für Ihr Ergebnis-Kapitel.

Der Schlussteil fasst Ihr Vorgehen und die Ergebnisse noch einmal zusammen.
Dabei nehmen Sie Bezug auf die Einleitung und die vorab gesetzten Ziele.
Am Ende des Schlussteils können Sie Ihre gesamte Arbeit nochmal rückblickend diskutieren, eventuelle Verbesserungsvorschläge aufzeigen und einen Ausblick auf mögliche, zukünftige Entwicklungen geben.
\end{enumerate}

Im Anhang können Sie ihrer Arbeit weitere Informationen hinzufügen, die nicht mehr zum eigentlichen Rahmen der Ausarbeitung passen.
Wenn Sie im Zuge ihrer Arbeit Dokumente entwickelt haben, die für den Leser interessant sind, können Sie diese in den Anhang aufnehmen.
Typische Beispiele sind Installationsanleitungen oder der Quellcode wichtiger Programmteile.
Bitte beachten Sie, dass der Anhang für eine mögliche Seitenvorgabe der Ausarbeitung nicht mitzählt.

\section{Form der Arbeit}

Natürlich steht der Inhalt der Arbeit an erster Stelle, aber ohne eine entsprechende Präsentation kann der Leser diese Inhalte nicht erschließen.
Daher sollte Ihre Ausarbeitung auch bestimmten formalen und ästhetischen Anforderungen genügen.

Weitestgehend festgelegt ist der strukturelle Aufbau einer Ausarbeitung.
Das \textbf{Titelblatt} enthält folgende Angaben:
\begin{compactitem}
\item Institution (z.B. \textit{Fachhochschule Südwestfalen})
\item Art der Arbeit (z.B. \textit{Bachelorarbeit}, \textit{Masterarbeit} oder \textit{Ausarbeitung})
\item Titel der Arbeit
\item Namen aller Autoren (bei Abschlussarbeiten können auch weitere Angaben erforderlich sein, z.B. Matrikelnummer, Geburtsdatum, Anschrift, \dots)
\item Datum (konkreter oder offizieller Abgabetermin)
\item Namen der Betreuer bzw. Referenten
\end{compactitem}

Bei Bachelor- und Masterarbeiten folgt auf das Deckblatt ein Abstract, optional zusätzlich in englischer Übersetzung.
Bei Hausarbeiten ist ein Abstract nicht unbedingt erforderlich.

Das nächste Hauptelement der Ausarbeitung ist das \textbf{Inhaltsverzeichnis}.
Hier sollten Sie darauf achten, dass Unterkapitel nur bis zu einer gewissen Ebene aufgenommen werden.
In aller Regel sollte die Gliederungszahl maximal drei- bis vierstellig (bei langen Dokumenten) sein.
Wenn Sie eine Danksagung in Ihre Arbeit aufnehmen wollen, sollten Sie diese zwischen dem Abstract und dem Inhaltsverzeichnis einfügen.

Es folgen die inhaltlichen Kapitel wie in Abschnitt~\ref{sec:gliederung} beschrieben.
Nach dem Schlussteil folgt als weiteres zentrales Element jeder wissenschaftlichen Ausarbeitung das \textbf{Literaturverzeichnis} (vgl. Abschnitt~\ref{sec:zitieren}).
Es folgen optionale weitere Verzeichnisse, ein Index und die Anhänge.
Für jede studentische Arbeit, die eine Prüfungsleistung darstellt, ist eine \textbf{Selbstständigkeitserklärung} notwendig, die üblicherweise ganz am Ende des Dokuments steht.

Die folgende Aufzählung zeigt die Struktur einer Ausarbeitung nochmal im Überblick.
Die Aufteilung in 6 Kapitel und die Überschriften sind dabei exemplarisch und können gegebenenfalls abweichen:
\begin{compactitem}
\item Deckblatt
\item Abstract
\item Inhaltsverzeichnis
\item[1] Einleitung
\item[2] Stand der Technik
\item[3] Methodik
\item[4] Umsetzung
\item[5] Evaluation und Ergebnisse
\item[6] Zusammenfassung und Ausblick
\item Literaturverzeichnis
\item weitere Verzeichnisse, z.B. Tabellenverzeichnis, Abbildungsverzeichnis, Abkürzungsverzeichnis (optional)
\item Index (optional)
\item Anhänge (optional)
\item Selbstständigkeitserklärung
\end{compactitem}


\subsection{Umfang}

Zum Umfang zählen nur die Seiten von Einleitung, Hauptteil und Schluss, das Titelblatt, die Verzeichnisse und der Anhang zählen nicht zum Umfang.
Laut der Fachprüfungsordnungen der Informatik-Studiengänge an der Fachhochschule Südwestfalen gelten folgende Richtwerte für schriftliche Ausarbeitungen:
\begin{compactitem}
\item Hausarbeiten und schriftliche Ausarbeitungen: 10-15 Seiten
\item Bachelorarbeiten: etwa 30 Seiten
\item Masterarbeiten: etwa 80 Seiten 
\end{compactitem}

Eine Seite hat dabei jeweils etwa 50 Zeilen.
Sollte Ihre Arbeit also viele große Abbildungen enthalten oder, wegen Seitenumbrüchen einige nur knapp gefüllte Seiten, erhöht sich die erwartete Seitenzahl entsprechend.
Allgemein sollten Sie eher versuchen, die Seiten ihrer Arbeit möglichst gut zu füllen, auch wenn die Seitenzahl dann eher geringer ausfällt.
Die reine Anzahl der Seiten ist selbstverständlich kein Qualitätskriterium.
Die Angabe soll Ihnen aber dabei helfen, den erwarteten Umfang besser einschätzen zu können.

\subsection{Die Wahl des \textit{richtigen} Textsatzprogramms}

Viele Studierende stehen spätestens vor ihrer ersten längeren schriftlichen Ausarbeitung vor der Frage, mit welchem Programm sie die Arbeit am besten schreiben sollen.
Naheliegend ist hier ein Textverarbeitungsprogramm wie \textit{Microsoft Word} oder \textit{OpenOffice Writer} zu verwenden, eine Alternative dazu sind  Textsatzsysteme wie etwa LaTeX.
In jedem Fall sollten Sie Formatvorlagen einsetzen, um einen gleichmäßigen Textsatz (Seitenränder, Schriftarten, Textgröße, Aufzählungen, \dots) zu gewährleisten.

Persönlich kann ich Ihnen nur dazu raten, Ihre Ausarbeitungen (und ggf. weitere Dokumente)  mit LaTeX zu verfassen.
Im Vergleich zu WYSIWYG-Editoren wie \textit{Word} oder \textit{Writer} erfordert LaTeX zwar einen etwas höheren Lernaufwand.
Das Druckbild, die Organisation von Querverweisen, Fußnoten und Zitaten sowie das Setzen von mathematischen Formeln spricht allerdings eindeutig für das Textsatzsystem.
Spezielle Editoren, wie z.B. \texttt{TeXstudio} erleichtern den Einstieg und die Arbeit mit LaTeX (siehe Abbildung~\ref{fig:editor}).
 
\begin{figure}
\centering
\includegraphics[width=0.7\linewidth]{texstudio.png}
\caption{\LaTeX\ Editor \texttt{TeXstudio}}
\label{fig:editor}
\end{figure}

Eine interessante Alternative zu \textit{Word} und LaTeX sind Dokumente im \textit{Markdown}-Format.
Ähnlich wie bei LaTeX, werden in Markdown Dokumenten Annotationen hinzugefügt, um bestimmte Textelemente zu formatieren.
Für Überschriften werden z.~B. je nach Tiefe eine entsprechende Anzahl von \texttt{\#}-Zeichen vorangestellt, Aufzählungen werden durch Spiegelstriche (\texttt{-})  markiert, u.s.w.
Allgemein ist der so entstehende Text noch sehr gut lesbar.
Entsprechende Werkzeuge können Markdown in HTML-Format umwandeln und entsprechend formatiert anzeigen.
Außerdem existieren Konvertierungswerkzeuge (z.~B. Pandoc\footnote{\url{https://pandoc.org/} (abgerufen am 13.05.2020)}), die Markdown Dokumente in viele weitere Formate (PDF, LaTeX, doc, \dots) übersetzen können.

\subsection{Formatierung}

Für ein einheitliches Schriftbild sollten Sie das Layout Ihres Dokuments fest einstellen und einheitlich anwenden.
Schriftarten im Fließtext werden nur geändert, wenn dies der Hervorhebung dient.
Die Abstände sollten so gewählt werden, dass einzelne Textelemente gut voneinander unterscheidbar sind aber gleichzeitig nicht \glqq{}auseinandergerissen\grqq{} wirken.
Folgende Übersicht fasst die wichtigsten Formatierungseinstellungen zusammen:
\begin{compactitem}
\item Als Schriftgröße für den normalen Fließtext sollte 11 pt oder 12 pt gewählt werden.
\item Für den Fließtext muss eine Serifen-Schrift ausgewählt werden, z.B. \textit{Cambria} oder \textit{Times New Roman}. \textsf{Verwenden Sie keine serifenlosen Schriftarten (z.~B. \textit{Arial, Verdana, Calibri} ) für den Fließtext!} Für Überschriften sind serifenlose Schriften OK.
\item Eine sinnvolle Wahl für die Seitenränder ist: links 3 cm, rechts 2 cm, oben und unten je 2,5 cm. Das Papierformat ist DIN A4.
\item Wählen Sie einen einfachen Zeilenabstand.
\item Wenn Sie Blocksatz verwenden, sollte unbedingt Silbentrennung eingestellt werden. Ansonsten weist das Schriftbild zu große Lücken auf und wirkt unschön.
\item Eine Seitennummerierung ist erforderlich.
\end{compactitem}

\subsection{Graphische Elemente}


Schaubilder und Diagramme schaffen Anschaulichkeit und sind häufig hilfreich um komplizierte Sachverhalte übersichtlich darzustellen.
Allerdings sollten Sie Ihre Abbildungen sorgfältig auswählen.
Die Faustregel hier ist, dass eine Abbildung immer zusätzliche Informationen liefern muss.
Ein Firmenlogo beispielsweise, ist eher ein Designelement, mit dem Sie einen Text auflockern können.
Bei wissenschaftlichen Texten sind solche Stilmittel aber unpassend und Sie sollten eher darauf verzichten.
Außerdem müssen Sie bei der Benutzung von nicht selbst erstellten Bildern das Copyright beachten.
Vielfach ist es nicht ohne weiteres gestattet, fremde Abbildungen in eigenen Werken zu übernehmen.
Wenn Sie Abbildungen verwenden, müssen diese zwingend mit einer Bildunterschrift versehen, und im Text referenziert werden.


\subsection{Literaturverzeichnis und korrektes Zitieren}
\label{sec:zitieren}

\begin{quote}
\textit{If you steal from one person it's plagiarism; if you steal from several it's called research.}
\vspace*{-4eX}$\ $
\begin{flushright}
    \small{--Wilson Mizner (1876–1933)}
\end{flushright}
\end{quote}

Das oben genannte Zitat beschreibt recht einleuchtend, dass in der Wissenschaft neue Erkenntnisse immer der Grundlage des bestehenden Wissens aufgebaut werden.
Beim Schreiben von wissenschaftlichen Texten muss daher immer erkennbar sein, welche Teile vom Autor geschaffen, und welche aus bestehenden Quellen entnommen wurden.
Das Übernehmen von Bestehendem ist dabei kein Manko, sondern unbedingt angebracht.
\textit{Sie müssen das Rad nicht neu erfinden!}
Für alle übernommenen Inhalte ist die Angabe der korrekten Quelle allerdings unablässig.
Sobald eine Quelle nicht oder unvollständig angegeben ist, muss man von einem Plagiat sprechen.
Eine solche Arbeit ist strenggenommen wertlos.
Sie sollten also beim Anfertigen Ihrer Arbeit strikt darauf achten, alle verwendeten Hilfsmittel anzugeben.
Andernfalls müssen Sie mit deutlichen Punktabzügen, in schweren Plagiatsfällen auch mit dem Nichtbestehen der Prüfung rechnen.

\paragraph{Quellen auswählen}

Bei der Auswahl von Hilfsmitteln sollte man beachten, dass verschiedene Quellen mehr oder weniger \textit{zitierfähig} sind.
Besonders zitierfähig sind Publikationen, die einem wissenschaftlichen Begutachtungsverfahren unterlegen sind.
Dies sind z.~B. Aufsätze, die Fachzeitschriften oder in Sammelbänden von Fachkonferenzen erschienen sind, oder auch Fachbücher von seriösen Verlagen.
Im Gegensatz dazu sind Internet-Quellen nicht in jedem Fall zitierfähig.
Daher müssen Sie genau darauf achten, dass die Quellen vertrauenswürdig sind und einem gewissen Anspruch genügen.
Näheres dazu, wie Sie entsprechende Literatur finden und auswählen, können Sie in Ratgebern zum wissenschaftlichen Arbeiten nachlesen (z.~B. in~\cite{Kremer2018} oder ~\cite{Kollmann2016}).

\paragraph{Zitieren}

Es gibt verschiedene Standards, nach denen Quellen im Text genannt und im Literaturverzeichnis aufgezählt werden.
Eine verbreitete Zitierweise ist der \textbf{Harvard-Stil}, bei dem als Kurznachweis im Text die Autoren, das Erscheinungsjahr des Werks und möglichst die Seitenangabe genannt wird  (vgl. Bergener \textit{et al.} 2019, S. 59).\nocite{Bergener2019}
Bei mehr als zwei Autoren verwendet man nur den Erstautor mit der Ergänzung \textit{et al.}
Unter dem Begriff \textbf{Deutsche Zitierweise} versteht man das Zitieren über Fußnoten.\footnote{\textit{Theisen, Manuel René}, Wissenschaftliches Arbeiten -- Erfolgreich bei Bachelor- und Masterarbeit, 16. Aufl., Franz Vahlen, 2013}\nocite{Theisen2013}
Bei einem sogenannten \textit{Vollbeleg} wird das gesamte Werk inklusive aller Autoren, Erscheinungsjahr und Werk angegeben.

Die beiden genannten Zitierweisen sind in den Wirtschaftswissenschaften bzw. den Geistes- und Sozialwissenschaften sehr verbreitet.
In den Ingenieurwissenschaften und der Informatik ist der \textbf{IEEE-Stil}~\cite{ieeestyle} vorherrschend.
Als Kurznachweis im Text wird dabei lediglich eine Zahl angegeben, die auf die eigentliche Quellenangabe im Literaturverzeichnis verweist.
Die beiden wichtigsten Fachgesellschaften der Informatik, IEEE und ACM, verwenden in ihren Publikationen diesen Zitierstil.

Für das Literaturverzeichnis gelten ebenfalls je nach Zitierstil verschiedene Richtlinien.
Dabei legt die Art der Publikation (z.~B. Fachbuch, Fachaufsatz in einem Journal oder in einem Tagungsband einer Konferenz, Handbuch oder technischer Bericht) fest, welche Angaben erforderlich sind.
Am einfachsten ist es, Sie verwenden ein Werkzeug wie BibTeX, mit dem Literaturverzeichnisse für verschiedene Stilvorlagen automatisch erzeugt werden können.\footnote{\url{www.bibtex.org} (abgerufen am 14.05.2020)} 

Welchen Zitierstil Sie für Ihre Ausarbeitung verwenden wollen, können Sie selbst entscheiden.
Wichtig ist es aber, dass Sie den Stil durchgängig anwenden.
Keine, fehlerhafte oder unseriöse Literaturangaben führen in aller Regel zu Punktabzügen, die sich (je nach Umfang) auch in der Endnote widerspiegeln.

\subsection{Ausdruck und Sprache}

Ausdruck und Sprache sollten einer wissenschaftlichen Ausarbeitung angemessen sein.
Vermeiden Sie Umgangssprache und verzichten Sie auf Füllwörter und eine zu ausschweifende Ausdrucksweise.
Ideal ist es, wenn Sie einen argumentativen Stil verwenden.
Benennen Sie möglichst klar, welche Entscheidungen Sie treffen (These) und wie Sie zu dieser Entscheidung kommen (Argument).
Mit passenden Beispielen oder Belegen können Sie ihre Argumentation stützen.
Wenn Sie beispielsweise im Rahmen Ihrer Arbeit ein Programm entwickeln, müssen Sie an vielen Stellen Designentscheidungen treffen.
Erläutern Sie diese Entscheidungen in der Ausarbeitung, indem Sie z.~B. angeben, warum sie eine bestimmte Alternative implementieren, oder warum Sie bestimmte Funktionalitäten weglassen.

Wissenschaftliche Texter sollen objektiv und allgemeingültig ein.
Daher ist die Verwendung der \textbf{Ich-Form} in den allermeisten Fällen unangemessen.
Versuchen Sie möglichst, Sachverhalte mit passiven oder unpersönlichen Formulierungen darzustellen.
Eine Ausnahme von dieser Regel sind Stellen im Text, an denen Sie Ihre persönliche Meinung zum Ausdruck bringen oder eigene Begriffe einführen:
\begin{compactitem}
\item \textit{\glqq{}Ich bin der Überzeugung, dass meine Methode hilfreich sein wird, um\dots\grqq{}}
\item \textit{\glqq{}Ich bezeichne dieses Verfahren im Folgenden als\dots\grqq{}}
\end{compactitem}
Dabei ist auch in diesen Fällen die Verwendung der 1. Person nur eine optionales Stilmittel.
In der Regel kann man die komplette Arbeit ohne die Ich-Form auskommen.


\textbf{Abkürzungen} werden bei der ersten Verwendung im Text immer ausgeschrieben und damit eingeführt.
Einzige Ausnahme sind allgemein bekannte Abkürzungen, die Sie nicht einführen müssen.
Bei längeren Texten mit vielen Abkürzungen 	ist ein zusätzliches Abkürzungsverzeichnis hilfreich.


%Gibt es beispielsweise für ein Programmierproblem mehrere alternative Bibliotheken, geben Sie an, warum Sie eine bestimmte auswählen.


\subsection{Rechtschreibung und Grammatik}

Es wird erwartet, dass eine schriftliche Ausarbeitung grammatikalisch und orthografisch korrekt ist.
Vereinzelte Fehler sind sicher nicht dramatisch.
Schwerwiegende und sich häufende Rechtschreib- oder Grammatikfehler führen aber üblicherweise zu Punktabzug.
Vor allem für Abschlussarbeiten gilt daher, dass Sie, und möglichst auch andere Personen,  Ihren Text Korrektur lesen sollten.


%Ein Buch wird zitiert~\cite{Theisen2013, Kremer2018, Bergener2019}


